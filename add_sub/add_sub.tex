\documentclass[a4paper,12pt]{article}
\usepackage{amsmath}
\usepackage{amsfonts}
\usepackage{amssymb}
\usepackage{graphicx}
\usepackage{geometry}
\usepackage{nopageno}

\pagestyle{empty}
\geometry{a4paper, margin=0.5in}


\begin{document}



\section*{Objective}
The objective of this lab is to implement a C program to perform binary addition and subtraction of two decimal numbers, including handling of negative numbers using two's complement representation.

\section*{Algorithm}

\begin{enumerate}
    \item Start.
    \item Initialize and take number inputs in decimal.
    \item Initialize arrays \texttt{first} and \texttt{second} of a specific bit size.
    \item Convert the first decimal number to binary. If the number is negative, compute the two's complement for the number and store it in \texttt{first}.
    \item Convert the second decimal number to binary. If the number is negative, compute the two's complement for the number and store it in \texttt{second}.
    \item Initialize two arrays \texttt{sum} and \texttt{diff} to store the sum of two numbers and their differences respectively.
    \item Add two numbers \texttt{first} and \texttt{second} and store the result in array \texttt{sum}. Print the output.
    \item Compute the two's complement of the number in the \texttt{second} array and add it to the \texttt{first} array. Store the output in array \texttt{diff} and print the output.
    \item Stop.
\end{enumerate}

\section*{Source Code}

\begin{verbatim}
#include <stdio.h>
#include <stdlib.h>

#define SIZE 5

void decimalToBinary(int n, int *binary, int size) {
    if (n < 0) {
        n = abs(n);
    }

    for (int i = size - 1; i >= 0; i--) {
        binary[i] = n % 2;
        n = n / 2;
    }
}

void twosComplement(int *binary, int size) {
    int carry = 1;

    for (int i = 0; i < size; i++) {
        binary[i] = binary[i] == 0 ? 1 : 0;
    }

    for (int i = size - 1; i >= 0; i--) {
        binary[i] = binary[i] + carry;
        if (binary[i] == 2) {
            binary[i] = 0;
            carry = 1;
        } else {
            carry = 0;
        }
    }
}

void addTwoBinaries(int *binary1, int *binary2, int size) {
    int carry = 0;
    for (int i = size - 1; i >= 0; i--) {
        binary1[i] = binary1[i] + binary2[i] + carry;
        if (binary1[i] == 2) {
            binary1[i] = 0;
            carry = 1;
        } else if (binary1[i] == 3) {
            binary1[i] = 1;
            carry = 1;
        } else {
            carry = 0;
        }
    }
}

void subtractTwoBinaries(int *binary1, int *binary2, int size) {
    int twosComp[SIZE];
    
    for (int i = 0; i < size; i++) {
        twosComp[i] = binary2[i];
    }
    
    twosComplement(twosComp, size);
    addTwoBinaries(binary1, twosComp, size);
}

void printBinary(int *binary, int size) {
    for (int i = 0; i < size; i++) {
        printf("%d", binary[i]);
    }
    printf("\n");
}

int main() {
    int a1, b1, a[SIZE], b[SIZE];

    printf("Enter the first number: ");
    scanf("%d", &a1);
    printf("Enter the second number: ");
    scanf("%d", &b1);
    printf("\n");

    decimalToBinary(a1, a, SIZE);
    if (a1 < 0) {
        twosComplement(a, SIZE);
    }

    printf("First number in binary: ");
    printBinary(a, SIZE);

    decimalToBinary(b1, b, SIZE);
    if (b1 < 0) {
        twosComplement(b, SIZE);
    }

    printf("Second number in binary: ");
    printBinary(b, SIZE);
    printf("\n");

    int sum[SIZE];
    for (int i = 0; i < SIZE; i++) {
        sum[i] = a[i];
    }
    addTwoBinaries(sum, b, SIZE);

    printf("Sum of the given two numbers: ");
    printBinary(sum, SIZE);

    int diff[SIZE];
    for (int i = 0; i < SIZE; i++) {
        diff[i] = a[i];
    }
    subtractTwoBinaries(diff, b, SIZE);

    printf("Difference of the given two numbers: ");
    printBinary(diff, SIZE);

    return 0;
}
\end{verbatim}

\section*{Sample Input/Output}
\begin{itemize}
    \item \textbf{Input:} 7 and -3
    \item \textbf{Output:}
    \begin{verbatim}
Enter the first number: 7
Enter the second number: -3

First number in binary: 00111
Second number in binary: 11101

Sum of the given two numbers: 00010
Difference of the given two numbers: 10100
    \end{verbatim}
\end{itemize}

\section*{Discussion}
The program successfully converts decimal numbers to binary, performs addition and subtraction using two's complement for negative numbers, and displays the results in binary format. The approach ensures correct handling of negative numbers by using two's complement, which simplifies binary arithmetic operations.

\section*{Conclusion}
The implemented C program correctly performs binary addition and subtraction of two decimal numbers, including handling negative numbers using two's complement representation. This lab demonstrates the process of binary arithmetic operations and the importance of two's complement in representing negative numbers.

\end{document}
