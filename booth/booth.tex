\documentclass[a4paper,12pt]{article}
\usepackage{amsmath}
\usepackage{amsfonts}
\usepackage{amssymb}
\usepackage{graphicx}
\usepackage{geometry}
\usepackage{array}
\usepackage{nopageno}
\geometry{a4paper, margin=0.5in}
\pagestyle{empty}

\begin{document}

\section*{Objective}
The objective of this lab is to implement a C program that multiplies two decimal numbers using Booth's algorithm, handling negative numbers using two's complement representation.

\section*{Algorithm}
\begin{enumerate}
    \item Start.
    \item Initialize and take number inputs in decimal.
    \item Initialize arrays of specific bit size \(Q\), \(M\) and \(M_{\text{comp}}\). Also, initialize accumulator to 0, an integer \(a = 0\), and counter = number of bits.
    \item Convert the first decimal number to binary. If the number is negative, compute the two’s complement for the number and store it in \(Q\).
    \item Convert the second decimal number to binary. If the number is negative, compute the two’s complement for the number and store it in \(Q\).
    \item Find the two’s complement of \(M\) and store it in \(M_{\text{comp}}\).
    \item For \(n\) bits of data:
    \begin{itemize}
        \item if \(Q[n-1].a = 01\), perform \(A \leftarrow A + M\) and right shift accumulator, \(Q\), and \(a\).
        \item else if \(Q[n-1].a = 10\), perform \(A \leftarrow A - M\) and right shift accumulator, \(Q\), and \(a\).
        \item else right shift accumulator, \(Q\), and \(a\).
    \end{itemize}
    \item Count--.
    \item If count > 0, go to step 7.
    \item Stop.
\end{enumerate}

\section*{Source Code}

\begin{verbatim}
#include <stdio.h>
#include <stdlib.h>
#include <math.h>

void decimalToBinary(int n, int *binary, int size)
{
    n = abs(n);
    for (int i = size - 1; i >= 0; i--)
    {
        binary[i] = n & 1;
        n >>= 1;
    }
}

void twosComplement(int *binary, int size)
{
    int carry = 1;

    for (int i = 0; i < size; i++)
    {
        binary[i] = ~binary[i] & 1;
    }

    for (int i = size - 1; i >= 0; i--)
    {
        binary[i] += carry;
        if (binary[i] == 2)
        {
            binary[i] = 0;
            carry = 1;
        }
        else
        {
            carry = 0;
        }
    }
}

void arithmeticRightShift(int *binary1, int *binary2, int *a, int size)
{
    *a = binary2[size - 1];

    for (int i = size - 1; i > 0; i--)
    {
        binary2[i] = binary2[i - 1];
    }

    binary2[0] = binary1[size - 1];

    int msb = binary1[0];
    for (int i = size - 1; i > 0; i--)
    {
        binary1[i] = binary1[i - 1];
    }
    binary1[0] = msb;
}

void addTwoBinaries(int *binary1, const int *binary2, int size)
{
    int carry = 0;
    for (int i = size - 1; i >= 0; i--)
    {
        int sum = binary1[i] ^ binary2[i] ^ carry;
        carry = (binary1[i] & binary2[i]) | (binary2[i] & carry) | (carry & binary1[i]);
        binary1[i] = sum;
    }
}

void printBinary(const int *binary, int size)
{
    for (int i = 0; i < size; i++)
    {
        printf("%d", binary[i]);
    }
}

void printRow(int count, const int *accumulator, const int *temp, int a, const char *operation, int size)
{
    printBinary(accumulator, size);
    printf(" | ");
    printBinary(temp, size);
    printf(" |   %d   |   %d   | %s\n", a, count, operation);
}

int main()
{
    int x, y, a = 0, count;
    int size;

    printf("Enter the size: ");
    scanf("%d", &size);

    int *first = (int *)malloc(size * sizeof(int));
    int *second = (int *)malloc(size * sizeof(int));
    int *accumulator = (int *)calloc(size, sizeof(int));
    int *complementSecond = (int *)malloc(size * sizeof(int));
    int *temp = (int *)malloc(size * sizeof(int));

    printf("Enter the first number: ");
    scanf("%d", &x);
    printf("Enter the second number: ");
    scanf("%d", &y);

    decimalToBinary(x, first, size);
    if (x < 0)
    {
        twosComplement(first, size);
    }
    printf("First number in binary: ");
    printBinary(first, size);
    printf("\n");

    decimalToBinary(y, second, size);
    if (y < 0)
    {
        twosComplement(second, size);
    }
    printf("Second number in binary: ");
    printBinary(second, size);
    printf("\n\n");

    for (int i = 0; i < size; i++)
    {
        complementSecond[i] = second[i];
    }

    twosComplement(complementSecond, size);

    for (int i = 0; i < size; i++)
    {
        temp[i] = first[i];
    }

    count = size;
    printf("|   A   |   Q   | Q-1 | COUNT | Remarks\n");
    printf("|-------|-------|-----|-------|-----------\n");

    printRow(count, accumulator, temp, a, "Initialization", size);

    while (count > 0)
    {
        if ((temp[size - 1] == 0) && (a == 1))
        {
            addTwoBinaries(accumulator, second, size);
            printRow(count, accumulator, temp, a, "Addition", size);
            arithmeticRightShift(accumulator, temp, &a, size);
            count--;
            printRow(count, accumulator, temp, a, "Shift", size);
            printf("\n");
        }
        else if ((temp[size - 1] == 1) && (a == 0))
        {
            addTwoBinaries(accumulator, complementSecond, size);
            printRow(count, accumulator, temp, a, "Subtraction", size);
            arithmeticRightShift(accumulator, temp, &a, size);
            count--;
            printRow(count, accumulator, temp, a, "Shift", size);
            printf("\n");
        }
        else
        {
            arithmeticRightShift(accumulator, temp, &a, size);
            count--;
            printRow(count, accumulator, temp, a, "Shift", size);
            printf("\n");
        }
    }

    printf("Result after Booth's multiplication:\n");
    printBinary(accumulator, size);
    printBinary(temp, size);
    printf("\n");

    free(first);
    free(second);
    free(accumulator);
    free(complementSecond);
    free(temp);

    return 0;
}
\end{verbatim}

\section*{Sample Input/Output}

\begin{itemize}
    \item \textbf{Input:} 6 and -5
    \item \textbf{Output:}
    \begin{verbatim}
Enter the size: 5
Enter the first number: 6
Enter the second number: -5
First number in binary: 00110
Second number in binary: 11011

\end{verbatim}
\end{itemize}
\begin{tabular}{|c|c|c|c|c|}
\hline
\textbf{A} & \textbf{Q} & \textbf{Q-1} & \textbf{COUNT} & \textbf{Remarks} \\
\hline
00000 & 00110 & 0 & 5 & Initialization \\
\hline
11001 & 00110 & 0 & 4 & Subtraction \\
\hline
11100 & 00011 & 0 & 4 & Shift \\
\hline
10011 & 00011 & 0 & 3 & Subtraction \\
\hline
10011 & 00011 & 1 & 3 & Shift \\
\hline
01001 & 00011 & 1 & 2 & Addition \\
\hline
01001 & 10001 & 1 & 2 & Shift \\
\hline
10100 & 10001 & 1 & 1 & Subtraction \\
\hline
11010 & 01000 & 1 & 1 & Shift \\
\hline
11101 & 01000 & 1 & 0 & Shift \\
\hline
\end{tabular}

\section*{Result after Booth's Multiplication}
\begin{tabular}{cc}
\textbf{A} & \textbf{Q} \\
\hline
11101 & 01000 \\
\end{tabular}

\section*{Discussion}
The program demonstrates the Booth's algorithm for multiplication of two numbers, taking care of the sign by converting the numbers to their two's complement if negative. The algorithm handles the bitwise operations step by step, providing a clear multiplication process in binary format.

\end{document}
