\documentclass[a4paper,12pt]{article}
\usepackage{amsmath}
\usepackage{amsfonts}
\usepackage{amssymb}
\usepackage{graphicx}
\usepackage{geometry}
\usepackage{array}
\usepackage{nopageno}
\geometry{a4paper, margin=0.5in}
\pagestyle{empty}

\begin{document}

\section*{Objective}
The objective of this lab is to implement a C program that multiplies two decimal numbers using Booth's algorithm, handling negative numbers using two's complement representation.

\section*{Algorithm}
\begin{enumerate}
    \item Start.
    \item Initialize and take number inputs in decimal.
    \item Initialize arrays of specific bit size $Q$, $M$, and $M_{\text{comp}}$. Also, initialize the accumulator $A$ to 0, an integer $(Q-1)  = 0$, and set the counter to the number of bits.
    \item Convert the multiplicand (first decimal number) to binary. If the number is negative, compute its two’s complement and store it in $M$.
    \item Convert the multiplier (second decimal number) to binary. If the number is negative, compute its two’s complement and store it in $Q$.
    \item Compute the two’s complement of $M$ and store it in $M_{\text{comp}}$.
    \item For $n$ bits of data:
    \begin{itemize}
        \item If $Q[0](Q-1)$ is $01$, perform $A \leftarrow A + M$ and then perform an arithmetic right shift on $A$, $Q$, $(Q-1)$.
        \item Else if  $Q[0](Q-1)$ is $10$, perform $A \leftarrow A - M$ and then perform an arithmetic right shift on $A$, $Q$,$(Q-1)$.
        \item Else, perform an arithmetic right shift on $A$, $Q$, $(Q-1)$.
    \end{itemize}
    \item Decrement the counter by 1.
    \item If the counter is greater othan 0, go to step 7.
    \item Stop. The result is stored in $AQ$.
\end{enumerate}

\section*{Source Code}

\begin{verbatim}
    #include <stdio.h>
    #include <stdlib.h>
    #include <math.h>
    
    void decimalToBinary(int n, int *binary, int size)
    {
        n = abs(n);
        for (int i = size - 1; i >= 0; i--)
        {
            binary[i] = n & 1;
            n >>= 1;
        }
    }
    
    void twosComplement(int *binary, int size)
    {
        int carry = 1;
    
        for (int i = 0; i < size; i++)
        {
            binary[i] = ~binary[i] & 1;
        }
    
        for (int i = size - 1; i >= 0; i--)
        {
            binary[i] += carry;
            if (binary[i] == 2)
            {
                binary[i] = 0;
                carry = 1;
            }
            else
            {
                carry = 0;
            }
        }
    }
    
    void arithmeticRightShift(int *binary1, int *binary2, int *a, int size)
    {
        *a = binary2[size - 1];
    
        for (int i = size - 1; i > 0; i--)
        {
            binary2[i] = binary2[i - 1];
        }
    
        binary2[0] = binary1[size - 1];
    
        int msb = binary1[0];
        for (int i = size - 1; i > 0; i--)
        {
            binary1[i] = binary1[i - 1];
        }
        binary1[0] = msb;
    }
    
    void addTwoBinaries(int *binary1, const int *binary2, int size)
    {
        int carry = 0;
        for (int i = size - 1; i >= 0; i--)
        {
            int sum = binary1[i] ^ binary2[i] ^ carry;
            carry = (binary1[i] & binary2[i]) | (binary2[i] & carry) | (carry & binary1[i]);
            binary1[i] = sum;
        }
    }
    
    void printBinary(const int *binary, int size)
    {
        for (int i = 0; i < size; i++)
        {
            printf("%d", binary[i]);
        }
    }
    
    void printRow(int count, const int *accumulator, const int *temp, int a, const char *operation, int size)
    {
        printBinary(accumulator, size);
        printf(" | ");
        printBinary(temp, size);
        printf(" |   %d   |   %d   | %s\n", a, count, operation);
    }
    
    int main()
    {
        int x, y, a = 0, count;
        int size;
    
        printf("Enter the size: ");
        scanf("%d", &size);
    
        int *first = (int *)malloc(size * sizeof(int));
        int *second = (int *)malloc(size * sizeof(int));
        int *accumulator = (int *)calloc(size, sizeof(int));
        int *complementSecond = (int *)malloc(size * sizeof(int));
        int *temp = (int *)malloc(size * sizeof(int));
    
        printf("Enter the first number: ");
        scanf("%d", &x);
        printf("Enter the second number: ");
        scanf("%d", &y);
    
        decimalToBinary(x, first, size);
        if (x < 0)
        {
            twosComplement(first, size);
        }
        printf("First number in binary: ");
        printBinary(first, size);
        printf("\n");
    
        decimalToBinary(y, second, size);
        if (y < 0)
        {
            twosComplement(second, size);
        }
        printf("Second number in binary: ");
        printBinary(second, size);
        printf("\n\n");
    
        for (int i = 0; i < size; i++)
        {
            complementSecond[i] = second[i];
        }
    
        twosComplement(complementSecond, size);
    
        for (int i = 0; i < size; i++)
        {
            temp[i] = first[i];
        }
    
        count = size;
        printf("|   A   |   Q   | Q-1 | COUNT | Remarks\n");
        printf("|-------|-------|-----|-------|-----------\n");
    
        printRow(count, accumulator, temp, a, "Initialization", size);
    
        while (count > 0)
        {
            if ((temp[size - 1] == 0) && (a == 1))
            {
                addTwoBinaries(accumulator, second, size);
                printRow(count, accumulator, temp, a, "Q[0]Q-1=10\tAddition A=A+M", size);
                arithmeticRightShift(accumulator, temp, &a, size);
                count--;
                printRow(count, accumulator, temp, a, "Arithematic Right Shift A Q Q-1", size);
                printf("\n");
            }
            else if ((temp[size - 1] == 1) && (a == 0))
            {
                addTwoBinaries(accumulator, complementSecond, size);
                printRow(count, accumulator, temp, a, "Q[0]Q-1=01\tSubtraction A=A-M", size);
                arithmeticRightShift(accumulator, temp, &a, size);
                count--;
                printRow(count, accumulator, temp, a, "Arithematic Right Shift A Q Q-1", size);
                printf("\n");
            }
            else
            {
                arithmeticRightShift(accumulator, temp, &a, size);
                count--;
                printRow(count, accumulator, temp, a, "Arithematic Right Shift A Q Q-1", size);
                printf("\n");
            }
        }
    
        printf("Result after Booth's multiplication:\n");
        printBinary(accumulator, size);
        printBinary(temp, size);
        printf("\n");
    
        free(first);
        free(second);
        free(accumulator);
        free(complementSecond);
        free(temp);
    
        return 0;
    }
    
\end{verbatim}

\section*{Sample Input/Output}

\begin{itemize}
    \item \textbf{Input:} 6 and -5
    \item \textbf{Output:}
    \begin{verbatim}
Enter the size: 5
Enter the first number: 6
Enter the second number: -5
First number in binary: 00110
Second number in binary: 11011

\end{verbatim}
\end{itemize}
\begin{tabular}{|>{\centering\arraybackslash}m{2cm}|>{\centering\arraybackslash}m{2cm}|>{\centering\arraybackslash}m{1cm}|>{\centering\arraybackslash}m{1.5cm}|>{\centering\arraybackslash}m{5cm}|}
    \hline
    \textbf{A} & \textbf{Q} & \textbf{Q-1} & \textbf{COUNT} & \textbf{Remarks} \\ 
    \hline
    00000 & 01001 & 0 & 5 & Initialization \\ 
    \hline
    00111 & 01001 & 0 & 5 & Q[0]Q-1=01 Subtraction A=A-M \\ 
    \hline
    00011 & 10100 & 1 & 4 & Arithmetic Right Shift A Q Q-1 \\ 
    \hline
    11100 & 10100 & 1 & 4 & Q[0]Q-1=10 Addition A=A+M \\ 
    \hline
    11110 & 01010 & 0 & 3 & Arithmetic Right Shift A Q Q-1 \\ 
    \hline
    11111 & 00101 & 0 & 2 & Arithmetic Right Shift A Q Q-1 \\ 
    \hline
    00110 & 00101 & 0 & 2 & Q[0]Q-1=01 Subtraction A=A-M \\ 
    \hline
    00011 & 00010 & 1 & 1 & Arithmetic Right Shift A Q Q-1 \\ 
    \hline
    11100 & 00010 & 1 & 1 & Q[0]Q-1=10 Addition A=A+M \\ 
    \hline
    11110 & 00001 & 0 & 0 & Arithmetic Right Shift A Q Q-1 \\ 
    \hline
    \end{tabular}

\section*{Result after Booth's Multiplication}
\begin{tabular}{cc}
\textbf{A} & \textbf{Q} \\
\hline
11101 & 01000 \\
\end{tabular}

\section*{Discussion}
The program demonstrates the Booth's algorithm for multiplication of two numbers, taking care of the sign by converting the numbers to their two's complement if negative. The algorithm handles the bitwise operations step by step, providing a clear multiplication process in binary format.

\end{document}
